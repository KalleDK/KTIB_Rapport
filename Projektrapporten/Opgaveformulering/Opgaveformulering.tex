%!TEX root = ../main.tex
\chapter{Opgaveformulering}
For at en gartner kan få det optimale udbytte af sine afgrøder, dvs. både mængde og kvalitet, kræver det, at gartneren er i stand til at give planterne de mest optimale vilkår. Det indebærer bl.a. vanding i tilpas mængder med tilpas mængder gødning tilsat. For et større gartneri kan det være et enormt arbejde at skulle overvåge jordens fugtighed, blande gødning samt at gå rundt og vande alle planterne. Man kan spare gartneren for hele dette arbejde ved at automatisere processen.
\newline
\newline
I dette projekt skal der laves et system, som automatisk kan gøde og vande planter. Dette medfører at systemet skal kunne:
\begin{itemize}
\item Blande gødning i et vandkar.
\item Dosere tilpas mængder vand fra karret til planterne.
\item Opretholde en tilpas pH-værdi af vandet i karret.
\end{itemize}

En sensor måler fugtigheden i jorden og dosere vand herudfra, så jorden altid har en tilpas fugtighed.
\newline
\newline
Der skal implementeres en grafisk brugergrænseflade i systemet, så også folk uden sans for IT kan benytte det. I brugergrænsefladen skal brugeren kunne indstille mængden af nogle forskellige slags gødninger, pH-værdi og jordfugtighed. Brugeren skal også have mulighed for at oprette og genbruge planer for, hvor meget gødning planterne skal have på bestemte tidspunkter i deres vækststadier.
\newline
\newline
I det færdige produkt er der fokus på at der er en præcis dosering af vand og gødning til planterne, der muliggør optimale livsvilkår for planter. Dertil skal brugervenligheden være i top, i form af et let anvendeligt grafisk interface.