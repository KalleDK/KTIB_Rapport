%!TEX root = ../main.tex
\chapter{Resumé}
\textbf{Dansk}
\\
AVS (Automatisk VandingsSystem) har til formål at automatisere vanding og gødning af planter i gartnerier. Et automatiseret system medfører at der skal bruges færre arbejdstimer og at arbejdet udføres ens hver gang. Dette resulterer i bedre afgrøder til billigere penge. Systemet er udarbejdet således, at det fungerer til små såvel som store gartnerier. Hele systemet kan styres fra en central computer. Hertil kobles et kar som indeholder vand med den rigtige mængde gødning. Hertil kobles en ø som opsamler data fra plantejorden. Kommunikation imellem de 3 resulterer i et fuldt funktionelt og automatiseret system. Til at udvikle systemet er der brugt en let version af Scrum. Iterationerne er kørt i 2-4 ugers intervaller og der har i flere af iterationerne været daglige møder. Alt dette har resulteret i en funktionel prototype med få mangler. For at færdiggøre prototypen skal dosering af gødning og intelligent håndtering af opsamlede data i forhold til vanding implementeres. Den udviklede prototype har givet anledning til flere forbedringer til et bedre produkt; hovedsagligt at samle HW-delene på færre print, implementere en bedre kabelføring og tilføje error-handling i SW.
\\\\
\textbf{English}
\\
The purpose of AVS (Automatic WateringSystem) is to automate watering and fertilizing of plants in a horticulture. Implementing an automated system results in fewer man-hours and a more identical execution of the desired task. This results in better crops for less money. The system is developed so it scales with small aswell as large horticultures. The entire system can be controlled by a central computer. Connected to this is a reservoir which contains water with the correct amount of fertilizer. Connected to this is a so-called \emph{Sensor Island} which collects data from the soil. Communication between these three systems results in a fully functionel and automated system. A light version of Scrum have been used to manage the project. The iterations have been divided into 2-4 week intervals. Several of the intervals had daily meetings. All this have resulted in a functional prototype with few deficiencies. To complete the prototype the dosage of fertilizer and intelligent handling of collected data relative to watering must be implemented. The prototype has sparked ideas for improvements to better the system; mainly assembling all the HW-parts on fewer PCBs, implementing better cable management and adding error-handling to the software.