%!TEX root = ../../main.tex
\section{Opnåede Erfaringer}
I opnåede erfaringer vil vi hver især reflektere på projektets forløb og hvad vi kan tage med derfra. De opnåede erfaringer i projektet vurderes invidividuelt, da hver projektdeltager har oplevet det på sin egen måde. Der kan forekomme ens opnåede erfaringer, men hvor lærerigt det har været kan være forskelligt fra projektdeltager til projektdeltager.

\subsection{Lærke}


\subsection{Kasper}


\subsection{Kalle}


\subsection{Jakob}


\subsection{Kenn}


\subsection{Karsten}


\subsection{Thomas}
I 3. semesterprojektet har jeg udover projektdokumentation arbejdet med PSU og RSConverter. I mit arbejde med hardware udvikling har jeg inddraget viden fra fagene EFYS, EEV, MSE og GFV. Strømforsyningen har været den største men bestemt også den mest interessante opgave jeg har arbejdet med. Jeg har haft mulighed for at udforske emner på egen hånd, som f.eks. regulering. Reguleringskredsløb er bygget op del for del, hvilket har medført mange ændringer undervejs, men også en god forståelse for hvad de forskellige fænomener skyldes.
\\\\
Projektet vi har valgt har været rigtig spændende. Forløbet fra idéen frem til prototypen har været meget lærerig. Den allervigtigste erfaring jeg tager med mig fra projektet er, hvor vigtigt det er med kommunikation imellem projektdeltagerne. En god kommunikation forhindrer misforståelser som senere hen kan skabe små såvel som store problemer.