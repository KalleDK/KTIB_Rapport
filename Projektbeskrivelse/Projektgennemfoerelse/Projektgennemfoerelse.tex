%!TEX root = ../../main.tex
\section{Projektgennemførelse}
Projektet startede med at vi skulle finde på en idé. Efter et par brainstormings fandt vi ud af, at det skulle være det automatiske vandingssystem. Herefter gik arbejdet ud på at finde ud af hvad det specifik skulle kunne. Da dette var besluttet blev der tegnet SysML diagrammer herunder oprettelse af vores usecases, BDD'er og IBD'er. Herefter uddelte vi nogle teknologiundersøgelser som gik ud på at finde ud af hvad der var den optimale løsning til den bedste pris. Undervejs holdte vi ugentlige møder hvor vi diskuterede vores opdagelser, samt kordinerede det med hvad der var muligt at realisere. Da vi havde fundet de enkelte dele fik vi dem bestilt og vi delte os op i hardware og software. Under hardwaren var det mest individuelt arbejde som blev lavet. Her var der tegnet SysML diagrammer og signaltabellerne blev lavet for at kunne specificere hvad der skulle modtages fra softwaresiden. Softwaren var mere handson, hvor der blev brugt viden fra faget ISU og det var derfor svært at gøre noget rigtigt første gang, da der hele tiden blev fundet en bedre måde at klare problemstillingen på. Da dette var meget tidskrævende blev der desværre ikke tid til at automatisere processerne. Derfor blev der kun implementeret manuel styring. Da undervisningen sluttede besluttede vi os for at samle vores enheder. Her fik vi på få dage samlet systemerne til et stort og efter en gennenført acceptest gik vi i gang med selve dokomentatione  og rapportskrivningen. 