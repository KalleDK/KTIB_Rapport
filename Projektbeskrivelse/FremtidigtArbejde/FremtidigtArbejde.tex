%!TEX root = ../../main.tex
\section{Fremtidigt Arbejde}
Projektforløbet har endt ud i en funktionel prototype. Prototypen gennemførte accepttesten og var et godt \emph{proof-of-concept}. Under dette afsnit vil vi diskutere det fremtidige arbejde som ville opstå i tilfælde af, at projektet skulle færdigudvikles. Det vil blive delt op i 3 punkter:
\begin{itemize}
\item Fremtidigt arbejde som indkluderer HW såvel som SW
\item HW
\item SW
\end{itemize}

\subsection{Fælles}
\begin{itemize}
\item 
\end{itemize}

\subsection{HW}
\subsubsection{Ventil-/pumpestyring}
\subsubsection{Flowsensor}
\subsubsection{Jordfugtsensor}
\subsubsection{pH-probe}
\subsubsection{PSU}
Strømforsyningsprototypen opfyldte deres funktion med små fejl og mangler. Spændingsreguleringen var opsat således, at spænding baserede sig på flere komponent som alle kommer med en tolerance. Dette gør at spænding vil variere fra produkt til produkt, og i værste tilfælde ikke være i stand til at opfylde dens funktion. Derudover var der ved hård belastning (ca. 50\% af den maksimale effekt), en tendens til at der opstod en 100Hz rippel på spændingen. Der er foreslået 2 ændringer til at løse disse problemer:

\begin{itemize}
\item Indføre flere elektrolytter på udgangen for at stabilisere spændingen ved hårde belastninger.
\item Indsætte et potentiometer i spændingsføleren som derved kunne indstille spændingen på udgangen.
\end{itemize}

\subsubsection{Shields til Raspberry og PSoC}

\subsection{SW} 
\begin{itemize}
\item 
\end{itemize}

